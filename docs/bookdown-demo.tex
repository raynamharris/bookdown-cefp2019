\documentclass[]{book}
\usepackage{lmodern}
\usepackage{amssymb,amsmath}
\usepackage{ifxetex,ifluatex}
\usepackage{fixltx2e} % provides \textsubscript
\ifnum 0\ifxetex 1\fi\ifluatex 1\fi=0 % if pdftex
  \usepackage[T1]{fontenc}
  \usepackage[utf8]{inputenc}
\else % if luatex or xelatex
  \ifxetex
    \usepackage{mathspec}
  \else
    \usepackage{fontspec}
  \fi
  \defaultfontfeatures{Ligatures=TeX,Scale=MatchLowercase}
\fi
% use upquote if available, for straight quotes in verbatim environments
\IfFileExists{upquote.sty}{\usepackage{upquote}}{}
% use microtype if available
\IfFileExists{microtype.sty}{%
\usepackage{microtype}
\UseMicrotypeSet[protrusion]{basicmath} % disable protrusion for tt fonts
}{}
\usepackage[margin=1in]{geometry}
\usepackage{hyperref}
\hypersetup{unicode=true,
            pdftitle={The Community Manager's Survival Guide},
            pdfauthor={The AAAS Community Engagment Fellows and Staff},
            pdfborder={0 0 0},
            breaklinks=true}
\urlstyle{same}  % don't use monospace font for urls
\usepackage{natbib}
\bibliographystyle{apalike}
\usepackage{longtable,booktabs}
\usepackage{graphicx,grffile}
\makeatletter
\def\maxwidth{\ifdim\Gin@nat@width>\linewidth\linewidth\else\Gin@nat@width\fi}
\def\maxheight{\ifdim\Gin@nat@height>\textheight\textheight\else\Gin@nat@height\fi}
\makeatother
% Scale images if necessary, so that they will not overflow the page
% margins by default, and it is still possible to overwrite the defaults
% using explicit options in \includegraphics[width, height, ...]{}
\setkeys{Gin}{width=\maxwidth,height=\maxheight,keepaspectratio}
\IfFileExists{parskip.sty}{%
\usepackage{parskip}
}{% else
\setlength{\parindent}{0pt}
\setlength{\parskip}{6pt plus 2pt minus 1pt}
}
\setlength{\emergencystretch}{3em}  % prevent overfull lines
\providecommand{\tightlist}{%
  \setlength{\itemsep}{0pt}\setlength{\parskip}{0pt}}
\setcounter{secnumdepth}{5}
% Redefines (sub)paragraphs to behave more like sections
\ifx\paragraph\undefined\else
\let\oldparagraph\paragraph
\renewcommand{\paragraph}[1]{\oldparagraph{#1}\mbox{}}
\fi
\ifx\subparagraph\undefined\else
\let\oldsubparagraph\subparagraph
\renewcommand{\subparagraph}[1]{\oldsubparagraph{#1}\mbox{}}
\fi

%%% Use protect on footnotes to avoid problems with footnotes in titles
\let\rmarkdownfootnote\footnote%
\def\footnote{\protect\rmarkdownfootnote}

%%% Change title format to be more compact
\usepackage{titling}

% Create subtitle command for use in maketitle
\providecommand{\subtitle}[1]{
  \posttitle{
    \begin{center}\large#1\end{center}
    }
}

\setlength{\droptitle}{-2em}

  \title{The Community Manager's Survival Guide}
    \pretitle{\vspace{\droptitle}\centering\huge}
  \posttitle{\par}
    \author{The AAAS Community Engagment Fellows and Staff}
    \preauthor{\centering\large\emph}
  \postauthor{\par}
      \predate{\centering\large\emph}
  \postdate{\par}
    \date{2019-05-25}

\usepackage{booktabs}
\usepackage{amsthm}
\makeatletter
\def\thm@space@setup{%
  \thm@preskip=8pt plus 2pt minus 4pt
  \thm@postskip=\thm@preskip
}
\makeatother

\begin{document}
\maketitle

{
\setcounter{tocdepth}{1}
\tableofcontents
}
\hypertarget{introduction}{%
\chapter{Introduction}\label{introduction}}

The \href{https://www.aaas.org/programs/community-engagement-fellows}{AAAS Community Engagement Fellows Program (CEFP)}, generously supported by the Alfred P. Sloan Foundation, aims to improve collaboration and community building in science. It provides a year-long professional development opportunity to individuals who cultivate member engagement and collaborative relationships within scientific associations, research collaborations and other communities of scientists. This book is a collection of blog posts that describe the experiences of many professional community managers. Their stories provide a wealth of knowlege that aspiring and practicing community managers can use.

\hypertarget{who}{%
\chapter{Who / what is a Community Manager?}\label{who}}

\emph{``Suddenly, my passion for open and transparent communication in science became `in-reach,' and I became a community manager. It was surprisingly empowering to finally have a tangible concept to assign to my new career path!''} - Elisha Wood-Charlson

A community engagement professional or community manager is someone who facilitates the activities of a community or collaboration and the interactions between its members. These activities may or may not take place within a dedicated online network, but likely involve the use of online tools. Community management may be considered as ``in-reach'' rather than ``outreach'' or public engagement. Typically, the role of a scientific community engagement manager (often part of the job description of a Center Director or Assistant Project Director) is to ensure cohesion in their community, yet they may have no formal training in the skills required to strategically do this. In science, the term community manager is gaining some traction within associations and within ``infrastructure'' organizations that serve scientists through the provision of specific tools and trainings. However, in a research collaboration context the term is rarely used and instead may be found within research development, program management, and research director roles.

\emph{``Can dedicated community management improve collaborative research? Help scientists make connections across disciplines and across institutions, regions, and nations? Spread expertise throughout associations and networks?''} - Dan Richman

In Section \ref{rationalCM}, we share data from a survey that explored why some organizations don't have community managers, examine the funding landscape for these roles, the educational backgrounds, and formal training that community managers have. We finish by introducing some of their top challenges and further training needs. In Section \ref{quickstart} provides a series of pieces with tips and tricks for new community managers, or those who have suddenly found themselves in such a role. Former Community Engagement Fellows share advice from their own experiences on topics including time management, setting priorities, networking, and how to work with stress.

\hypertarget{rationalCM}{%
\section{Rational for community managers}\label{rationalCM}}

\hypertarget{quickstart}{%
\section{Quick start}\label{quickstart}}

\hypertarget{what-makes-a-community}{%
\chapter{What makes a community?}\label{what-makes-a-community}}

\hypertarget{diversity-of-communities}{%
\section{Diversity of communities}\label{diversity-of-communities}}

\hypertarget{community-theory}{%
\section{Community theory}\label{community-theory}}

\hypertarget{tools-and-technologies}{%
\section{Tools and technologies}\label{tools-and-technologies}}

\hypertarget{putting-theory-into-practice}{%
\section{Putting theory into practice}\label{putting-theory-into-practice}}

\hypertarget{communities-in-action}{%
\section{Communities in action}\label{communities-in-action}}

\hypertarget{program}{%
\chapter{Programming and planning}\label{program}}

\hypertarget{creating-and-planning-events}{%
\section{creating and planning events}\label{creating-and-planning-events}}

\hypertarget{events-how-tos}{%
\section{events how-tos}\label{events-how-tos}}

\hypertarget{building-culture-inclusive-by-default}{%
\chapter{Building Culture: Inclusive By Default}\label{building-culture-inclusive-by-default}}

\hypertarget{cms-as-leaders}{%
\section{CMs as leaders}\label{cms-as-leaders}}

\hypertarget{diversity-equity-and-inclusion}{%
\section{Diversity, Equity, and Inclusion}\label{diversity-equity-and-inclusion}}

\hypertarget{conclusion}{%
\chapter{Conclusion}\label{conclusion}}

We have finished a nice book.

\bibliography{book.bib,packages.bib}


\end{document}
